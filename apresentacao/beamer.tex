%%% もし pdfTeX や LuaTeX を使うなら dvipdfmx オプションを外す.
% \documentclass[dvipdfmx]{beamer}
\documentclass{beamer}
\usepackage[utf8]{inputenc}
\usepackage[portuguese]{babel}
\usepackage{mathspec}
\usepackage{xeCJK}
\usepackage{graphicx}
\usepackage{url}
\usepackage{color}
\usepackage{multicol}
\usepackage[style=iso]{datetime2}
\setCJKmainfont{IPAPMincho}
\setCJKsansfont{IPAGothic}
\setCJKmonofont{IPAGothic}

% You can set fonts for Latin script here
\setmainfont{FreeSerif}
\setsansfont{FreeSans}
\setmonofont{Latin Modern Mono}

%%% もし pTeX + dvipdfmx を使うならば以下のどちらかを環境に合わせてコメントアウト.
%% \AtBeginDvi{\special{pdf:tounicode EUC-UCS2}} % EUC の場合
%% \AtBeginDvi{\special{pdf:tounicode 90ms-RKSJ-UCS2}} % SJIS の場合

%%% もし LuaTeX で日本語を出力するなら以下をコメントアウト.
%% \usefonttheme{luatexja}
%% \hypersetup{unicode}

%%% 日本語を使うなら以下を入れると定理環境中のフォントが立体になる.
%%% 欧文なら不要.
%%% LLT: Comment out this line if your presentation is in English or other European languages

\title{Driver do Shutdown Condicionado}
\author{Davi Vilarinho}
\institute{UFU \\ Faculdade de Computação \\ Universidade Federal de Uberlândia}
\date{\today}

\begin{document}

\begin{frame}
  \maketitle
\end{frame}

\begin{frame}{Relembrando...}
  Um \emph{device driver} é um código, normalmente escrito pelo fabricante, para
  controlar um dispositivo de hardware.

  Esse controle pode ser programado, interrupções ou DMA.

  Interrupções são eventos que ocorrem quando um sinal é emitido por um
  dispositivo e a CPU para o que está fazendo para tratar o ocorrido.
\end{frame}

\begin{frame}{Módulo do Kernel}
  Um módulo do Kernel não é nada mais que código dinamicamente carregado pelo
  kernel on the fly.

  \begin{block}{Compilação e Carregamento}

  O kernel é um executável, mas quando há compilação de um módulo, insere-se
  instruções ao executável existente, mas para isso um mecanismo de build 
  especial para módulos faz-se necessários.
  
  Garante-se acesso aos recursos do kernel pelo código carregado dinamicamente.
  \end{block}
\end{frame}

\begin{frame}{A estrutura de um módulo de kernel}
  Sem efeito para módulos, mas convenção:

  \begin{itemize}
      \item \texttt{\_\_init}: macro para subrotina que inicializa o módulo.
    
      \item \texttt{\_\_exit}: macro para subrotina que descarrega o módulo.
    
      \item \texttt{init\_module}: macro para indicar qual é a subrotina que inicializa o módulo.
    
      \item \texttt{exit\_module}: macro para indicar qual é a subrotina que descarrega o módulo.
  \end{itemize}
\end{frame}

\begin{frame}[fragile]{O código}
\begin{verbatim}
    #include <linux/module.h>
    ...
    static int __init inicio_do_seu_modulo(void) 
    ...
    static void __exit fim_do_seu_modulo(void)
    ...

    module_init(inicio_do_seu_modulo);
    module_exit(fim_do_seu_modulo);

    MODULE_LICENSE("GPL"); // ou a sua license
\end{verbatim}
\end{frame}

\begin{frame}[fragile]{A implementação da IRQ}
  Um \emph{device driver} utiliza de IRQ's e se registra para ser capaz de
  responder aos eventos por ele definidos.

\begin{verbatim}
  __init
  ...
  if (request_irq(IRQ_TECLADO, interrupcao_teclado, 
      IRQF_SHARED, "interrupcao_teclado", 
      (void *)&teclado_device)) {
  ...
\end{verbatim}
\end{frame}

\begin{frame}[fragile]{O interrupt handler}
\begin{verbatim}
static irqreturn_t interrupcao_teclado(int irq, 
  void *dev_id) {
  if (irq != 1)
    return IRQ_NONE;
  codigo_da_tecla_da_controladora_de_teclado = 
    inb(0x60);
  if (... não são teclas de interesse ...) {
    return IRQ_NONE;
  }
  queue_work(system_long_wq, 
    &check_shutdown_condition_task);
  return IRQ_HANDLED;
}
\end{verbatim}
\end{frame}

\begin{frame}{O interrupt handler}
  \begin{itemize}
    \item{\texttt{inb(ADDRESS)}: esta chamada busca o que está na porta
      enderaçada por "ADDRESS", neste caso porta é o mecanismo usado pela CPU
      para comunicar com o dispositivo. Para o dispositivo teclado, o dado da
      tecla pressioanda está em "0x60"}
    \item{179 e 180 se relacionam ao \emph{clique} das teclas ',' e '.',
      enquanto 51 e 52 o pressionamento das mesmas}
    \item{\texttt{queue\_work(FILA\_DO\_KERNEL, FILA\_DE\_TASKS)} aqui
      escolhe-se a fila destinadas para tasks maiores do Kernel a execução da
      checagem das condições de cliques e shutdown.}
  \end{itemize}
\end{frame}

\begin{frame}{Filas de Trabalho}
  Utilizando de uma estrutura do kernel \texttt{work\_struct}, é possível
  definir subrotinas para serem executadas por threads do kernel em nível
  privilegiado.

  No caso foi necessário porque o desligamento faria uma interrupção nunca ser
  finalizada, então tanto a checagem (por questão de performance e prioridade)
  quanto o desligamento são tarefas executadas numa fila de longa duração do
  kernel.
\end{frame}

\begin{frame}[fragile]{O desligamento e checagem}

\begin{verbatim}
static void check_shutdown_condition(
  struct work_struct *w) {
  if (codigo_da_tecla_da_controladora_de_teclado == 51) {
    pressionado = 1;
  } else if (pressionado &&
    (codigo_da_tecla_da_controladora_de_teclado == 180 ||
    codigo_da_tecla_da_controladora_de_teclado == 52)) {
    printk("shutdown detectado");
    kernel_power_off();
  } else {
    pressionado = 0;
  }
}
\end{verbatim}
\end{frame}

\begin{frame}{O desligamento e checagem}
  Existe a verificação do estado de pressionamento das teclas e a chamada da rotina de desligamento do sistema.

  Esta rotina executa subrotinas do kernel que preparam para o desligamento
  adequado e migram para o desligamento dos dispositivos e da máquina.
\end{frame}

\end{document}
\endinput

%%
%% End of file `example_DarkConsole.tex'.
