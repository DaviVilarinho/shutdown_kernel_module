% xelatex
\documentclass{article}

\usepackage[utf8]{inputenc}
\usepackage[portuguese]{babel}
\usepackage{fontspec}

\usepackage{ot-tableau}
\usepackage{listings}
\usepackage[backend=biber]{biblatex}
\usepackage{easylist}
\usepackage{hanging}
\usepackage{hyperref}
\usepackage{blindtext}
\usepackage{tipa}
\usepackage{cgloss4e}
\usepackage{gb4e}
\usepackage{qtree}
\usepackage{enumerate}
\usepackage{longtable}
\usepackage{textgreek}
\addbibresource{~/vilarinho/latex-templates/uni.bib}

\title{Device Driver: Shutdown Condicionado \\
      \small TCD Sistemas Operacionais com Rivalino Matias Júnior}
\author{Davi Vilarinho - 12011BCC006}

\begin{document}

\maketitle

\begin{abstract}
  Um \emph{device driver} é um código que roda em modo núcleo e processa
  os eventos e funcionalidades de um dispositivo acoplado ao sistema 
  computacional operante e, por meio de uma programação específica, fornece
  uma interface para as aplicações e o sistema operacional operá-lo de acordo
  com o esperado. No caso deste trabalho, foi registrada (mais uma) interrupção
  para o dispositivo teclado que só maneja cliques e pressionar as teclas '.'
  (ponto) e ',' (vírgula). Caso ',' seja segurado e clique '.', então o
  kernel registra uma tarefa de shutdown.
\end{abstract}

\section{Introdução}

De acordo com \textcite{TanenbaumBos14}, o sistema operacional tem duas
funções que são: forcener aos programadores e programas aplicativos recursos
abstratos formando uma interface de uso dos dispositivos de \emph{hardware} e gerenciar
esses recursos. O código do sistema operacional contém instruções para a
realização de ambas as funções e formam o \emph{Kernel} de um sistema operativo.

Dentro desse código, existem trechos que se destinam ao gerenciamento
e abstrações fornecidas às aplicações quanto aos subsistemas de
memória, arquivos, processos, entrada e saída e, para este o último, foco do
trabalho, têm-se \emph{drivers} que operam um dos dispositivos (ou que realizam
função similar).

Existem dois tipos de dispositivos, os de blocos e os de caracteres que
classificam os \emph{drivers} de dispositivos para \textcite{TanenbaumBos14},
o primeiro são aqueles em que podem ser endereçados independentemente e o
segundo onde existe o fluxo contínuo.

Assim como descrito nos materiais da disciplina de \textcite{Slide7Rivalino},
os dispositivos podem ter entrada e saída programada, a maneira mais simples
onde a CPU é ocupada e espera-se a conclusão da entrada e saída, com
\emph{polling} do dispositivo. No entanto, isto funciona melhor para
dispositivos com alta vazão onde o tempo ocioso da CPU é baixo, do contrário,
por mais que simples não é vantajoso. Outra forma é orientado à interrupções, 
onde o dispositivo interrompe a CPU, indicando que está pronto, e nesse
intervalo de tempo a CPU é liberado à outras atividades, mas ocorre um
tratamento específico para cada interrupção. E a última forma é a utilização de
DMA (\emph{Direct Memory Access}), que executa sem a presença da CPU e reduz a
quantidade de interrupções da CPU.

\subsection{Device Drivers}

Ainda de acordo com \textcite{TanenbaumBos14}, normalmente escrito pelo
fabricante, para realizar as funções já descritas, é necessário que o código
execute em modo núcleo, justamente para ter acesso aos registradores envolvidos
nos processos de entrada e saída de um determinado dispositivo e disparar
rotinas de \emph{kernel} necessária para o tratamento dos eventos envolvidos à
ele.

Uma das rotinas necessárias é registrar uma rotina à uma interrupção específica,
criando assim \emph{interrupt handlers}, e assim o código consegue tratar um
evento determinado do dispositivo e terminar seu processamento com o efeito
esperado do dispositivo, como um clique de um botão do mouse ser tratado e
espalhar aos demais níveis de aplicação.

\subsection{Interrupções e IRQ's}

Interrupções são sinais emitidos por um dispositivo para a CPU
\cite{interruptsOSDevWiki}.

\emph{Interrupt Handlers} formam a primeira camada de software sobre a qual
\emph{device drivers} dependem de acordo com \cite{TanenbaumBos14}. São trechos
de código que são executados quando uma interrupção ocorre. Neste caso, salva-se
o contexto do que quer que estivesse executando na CPU, carregar um contexto
específico para o tratamento, cria o \emph{overhead} operacional necessário para
tratá-lo (como a pilha), sinaliza ao controlador para reabilitar interrupções,
executa a rotina associada e continua a execução da CPU a partir do processo
algoritmicamente selecionado.

Isto implica que, assim como \emph{device drivers} podem ter eventos múltiplos
de um mesmo dispositivo ou não, é necessário que o código seja \textbf{reentrante},
ou seja, é necessário que o código possa parar a execução e tratar uma
interrupção (que poderia ser a própria rotina já executando). Caso não possa
(como um disco que está lentamente escrevendo um bit e precisa operar partes
mecânica antes de causar acidentes), podem ser usados semáforos ou mesmo
desabilitar interrupções.

No \emph{Kernel}, a rotina que registra um interrupt handler é a
\lstinline{request\_irq}, que pela \cite{LinuxDocsLinuxGenericIRQHandling}
recebe como argumentos o número identificador da IRQ, a subrotina que será
executada de tipo \lstinline{irq\_handler\_irq}, \emph{flags} (definidas em
\cite{interruptHSourceCode}, indicam condições de execução e tratamento de
eventos paralelos e como eles devem (ou não devem) impactar o \emph{handler}), o
nome da interrupção e o id do dispositivo. Caso tenha retorno diferente de zero
houve um erro.

Similarmente, remove-se uma irq com a subrotina \lstinline{free\_irq}, que recebe 
a IRQ e o dispositivo que tem a interrupt handler a ser removido. 

\subsubsection{Reentrância e }

\printbibliography

\end{document}
